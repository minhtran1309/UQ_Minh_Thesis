\chapter[Introduction]{Introduction}
\label{Chap:Intro}

% ***************************************************
% Introduction
% ***************************************************

\section{Motivation}

% ********* Enter your text below this line: ********
%Replace "Your thesis topic" with what will become the \label{} for this section.
Cell-cell communication is mechanisms that enable one cell to influence the behavior of itself or another cell to ultimately coordinate biological processes within a tissue. It is vital for individual cells to be able to communicate with each other and their environment as a part of a functional multicellular organism to enable higher-order biological processes. For example, during embryonic development, cells differentiate into complex tissues and organs and these fate decisions are controlled through communications with neighboring cells \cite{gale1996eph, eichmann1997ligand}. Responses of immune cells against pathogens or tumours cells is another example of the cell-cell communication. When these signals are missing (i.e. in vitro), certain cell types enter a suicide program known as apoptosis. 

It is natural to make an analogy between the cell signalling systems and a human social organization. Cells heavily rely on the interaction between individuals to be functional and would fall apart if communications are interrupted \cite{bartee2018principles}. The mechanisms behind cell communication commonly involve extracellular proteins called ligands and transmembrane proteins called receptors \cite{alberts2018molecular}. 

Through advances in molecular biology, we can now intensively characterise these signalling mechanisms and even simulate the cell-cell signalling networks to better understand the principles of how the interactions between cells can drive biological function \cite{sprinzak2010cis, teague2016synthetic, toda2019engineering}. Furthermore, technology development enables us to conduct more experiments in cells within living organisms \cite{helmchen2005deep, periasamy2013methods}, and spawns new means to study the cell microenvironment within tissue context. In particular, Visium (10x Genomics) for spatial transcriptomic sequencing of genes and CosMX SMI (NanoString) for high-plex spatial profiling of proteins will be described. Lastly, I will present existing computational methods to analyse these data and highlight the need to develop an analytical pipeline to integrate multi-omics data to study the interaction between cells within tissue context.      

% ***************************************************
\section{Literature review}
\subsection{Biology background of cellular communication}
Cell communication is important for the tissue or organ to remain functional. An intuitive  type of cell communication is contact-dependent interaction when two cells are physically in contact via \textit{gap junctions}. These are specialized intercellular connections that link the cytoplasm of two adjacent cells via narrow filled channels. The channels allow cells to exchange various small molecules, ions but not macromolecules, such as proteins or nucleic acids. Studies  show that communication through gap junctions plays crucial roles in cell synchronization, differentiation, embryonic development, and immune response \cite{white1999genetic, vinken2006connexins}. For example, severe defects in heart development were found in mouse embryos to be caused by the mutation of one particular gap-junction protein (connexin 43) \cite{huang1998gap}. 

Some cells secrete signalling molecules to self-regulate or to control other neighboring cells of the same type. This type of communication is called \textit{autocrine signalling}. In \textit{autocrine signalling} (or intracrine signalling), cells stimulate themselves by sending signal molecules and back to their own receptors.  Autocrine signalling is most effective when occurring within the same cell or with neighboring cells of the same type. The process is used to encourage groups of cells to make the same developmental decision. Thus, autocrine signalling is often found in cells that are differentiated along a particular pathway to reinforce developmental decisions \cite{alberts2018molecular}. Autocrine signalling is often found in cancer cells and enhances cell differentiation and proliferation \cite{sporn1985autocrine}.  

While gap junctions are contact-dependent communications and allow cells to exchange ions or second messengers, cell communication also can take place across near or long distances. Unlike autocrine signalling, which often occurs  between cells of the same type, all forms of distance signalling often occur between different cells of different types. In most cases, signal molecules are secreted and act as local mediators by a cell, affecting neighboring cells in the immediate environment, or transmitted through the circulation to act on target cells at distant sites \cite{cooper2004cell, alberts2018molecular}. The former process when molecules are transmitted via a distance is called \textit{paracrine signalling}. In paracrine signalling, cells secrete signalling molecules called ligands and induce changes in nearby cells. The ligand molecules in paracrine signalling tend to be rapidly absorbed by neighboring target cells; destroyed by extracellular enzymes, or immobilized by the extracellular matrix. Prominent examples of paracrine signalling systems include neurotransmitters between nerve cells at a synapse,  blood clotting system, tissue repair, and formation of scars. [...]

A second type of signalling molecules that are specialized for long distant communication is considered \textit{endocrine signalling}. Endocrine cells secrete their signal molecules in the form of hormones, into the blood circulation, which sends the signal to targets cell throughout the body. Signal transmission in endocrine systems is much slower than in paracrine signalling and tends to last for longer as it relies on diffusion and blood flow. One classic example is pancreatic cells producing the hormone insulin which signals cells in fat, muscles, and liver to absorb glucose and maintain blood sugar levels. [...]

It has been reported that cell behaviors which consist of different complex processes are governed by specific combinations of extracellular signals rather than a single signal alone. Each cell in a multicellular animal has been programmed to respond to a specific set of extracellular signals to survive and perform specialised functions. Even with the same signal, distinct cell types respond differently as the responses of the cells rely on the combination of receptor proteins that they possess. There are thousands of ligand-receptor pairs that have been curated over the last decade \cite{salwinski2004database, orchard2012protein} and the number of signalling combinations and responses is almost infinite. However the types of responses can be grouped into two categories: (1) changes in gene expression or levels of intracellular proteins; (2) changes in outward behavior or appearance such as cell differentiation, division, or death. By studying cell communication, we can systematically understand coordinated  cellular behaviors and unravel complex extracellular responses. [...]

In cancer pathology studies, cross talk between cancer cells and immune cells is also closely connected to the tumourigenesis process including: initiation, progression, and metastasis \cite{wang2017role}. The role of the immune system is to protect the host from infectious pathogens and remove damaged cells \cite{davis2007molecular}. In the early stage of cancer progression, immune cells secrete signalling molecules (paracrine or/and endocrine) to trigger a cell apoptosis process to eliminate cancer cells. If not all cells are completely cleared during this elimination stage, immune cells will distribute around the cancer cells to suppress tumour growth \cite{bronkhorst2011detection, ly2010aged}. At a later cancer stage, metastasis happens when the cancer cells can mutate and assume a phenotype that helps them to elude immune system surveillance. In another word, the continuous cross signalling between immune-cancer cells exerts selective pressure on the cancer cells become more motile \cite{giampieri2009localized,ilina2009mechanisms}. The growth of the tumour can be classified as the consequence of cross cell type communication and morphological dependence. As a result, modelling of immune-cancer cell interactions creates great opportunities for tumour management through the development of targeted therapy including immunotherapy.[...]

My thesis will focus on ligand-receptor communications between different cell types and the study of the ligand-receptor interactions in a cancer context for two reasons. Firstly, cancer is a complex and heterogeneous set of cells with diversity of interaction \cite{brucher2014cell}. Cross communication between tumour cells and other cell types, especially immune cells, plays essential roles in tumour development as well as the feedback of tumour treatment (i.e. targeted therapy, immunotherapy) \cite{hanahan2000hallmarks}. Secondly, the availability of technologies to dissect the tumour microenvironments and characterise the cell-cell communication is very diverse, yet the number of computational methods to predict and assess quantitatively cell-cell communication within tumours is still limited. At present, the emergence of targeted therapy and immunotherapy results in the need for more analysis methods to validate ligand-receptor candidates for cancer treatments. [...]

\subsection{The importance of cell-cell communication with spatial context}
The wide application of single-cell sequencing has identified rare cellular properties, molecular features of cell populations, and cell-cell heterogeneity. However, cell populations comprise multiple layers of signalling across cell types which are location-dependent, making them more heterogeneous than expected. Gene expressions in a cell are dictated not only by molecular profiles but also by space (i.e. its position in tissues and/or organs) and time (i.e. stage of cell cycle or developmental stage of the organ) \cite{salomon2020genomic}. Therefore, the ability to integrate single-cell molecular profile with spatial context in studying cell-cell communication can benefit research into the biology of development and disease.[...] 

In development, embryonic stem cells (ESCs) are able to grow (or differentiate) into all somatic cell types (more than 200 cell types). However, the differentiation process of ESCs is not arbitrary but is driven by multiple processes especially cell-to-cell signalling. Specifically, the distributions of signalling proteins within some areas of an embryo drive the gene expression of the local cells through cell communication (also known as morphogen gradients), which decides cell differentiation choices. In a cell concentration-dependent fashion, a protein is distributed in the form of spatial gradients along the main embroynic axes \cite{trisnadi2013image, ramel2013ventral}. Therefore, a model of cell-cell interaction and gene expression patterns with spatial context would offer a potential way to understand and therefore harness developmental programs.[...]  

Another area of study in which cell communication is greatly influenced by the spatial context of the tissue is about cell interaction in tumour microenvironments. Understanding heterogeneity within the tumour and stromal compartments remain essential in understanding the cancer development and evaluation of treatment progress \cite{pages2010immune}. The heterogeneity of the cells within tumour differ from each other in both physical features and gene expression. A unique ability of cancer cells is their quick adaptation to different environmental conditions, affecting cellular morphology in order to survive \cite{clark2015modes}. In addition, tumour growth and cancer invasion are the landmark events with location-dependent where a small number of local cells  grow into the tumour, start to metastasize and finally become life-threatening disease \cite{friedl2011cancer}. Studies have shown that there is a positive feedback loop between cell-cell interaction, the cancer cell migration and the tissue microenvironments; in which the increase in intratumour cell-cell interaction promotes collective migration in cancer \cite{friedl2011cancer, whiteside2008tumor}, leading to dynamically induced the disorganization of nearby microenvironments of the tissue\cite{friedl2012classifying, canel2013cadherin, almendro2013cellular, roussos2011chemotaxis, zervantonakis2012three}. Meanwhile, the presence of macrophages, which are usually found near blood vessels, in the microenvironment can also stimulate cancer cells to become more motile \cite{wyckoff2007direct}. It is often known that cancer originates from genetic instability. However, the growth of cancer cells into the tumour and the process of cancer metastasis are constrained by cellular interaction and tissue spatial conditions that dictate the interaction within and surround the cancer nest \cite{west2019cellular, liotta2001microenvironment,anderson2006tumor}. [...]

Realizing the significance of cell-cell communication, many computational methods have been proposed to infer the ligand-target links between cells such as CellPhoneDB \cite{efremova2020cellphonedb}, NicheNet \cite{browaeys2020nichenet}, SingleCellSignalR \cite{cabello2020singlecellsignalr}, NATMI \cite{hou2020predicting} and iTalk \cite{wang2019italk}. These methods have been  used to predict the possible ligand-receptor communication between single cell RNA sequencing data (scRNA-seq). They come with advantages and disadvantages which will be further discussed in section \ref{subsec:existing_ccc_methods}. Yet a common constraint in all of these tools is the lack of spatial context. As the result, for cancer study the results from these software packages become less favorable and insufficient \cite{de2020unraveling}. To study the complexity of cell communication in cancer, multidimensional analyses with scRNA-seq data and spatially-resolved data are paramount. By integrating spatial context into studying cell communication, it would greatly improve the justification of the ligand receptor inference result and satisfy the grasp for modelling intratumour heterogeneity and cell communication during tumour development \cite{crosetto2015spatially, pages2010immune, marusyk2012intra,bedard2013tumour}.[...] To be revised and edited  

\subsection{Preliminaries of methods for cell-cell communication}
\subsubsection{Experimental approaches}
In the past, intercellular communication was examined by measuring intercellular electrical signals in the cell membrane \cite{bennett1966physiology, loewenstein1967intercellular, de1982cell}. The presence of gap junctions allows cells to exchange a variety of ions and small molecules, thus they reflect the cell\'s electrical permeability properties \cite{penn1966ionic, bennett1966physiology,loewenstein1966permeability,loewenstein1974cellular}. By utilising such electrical properties of membrane junctions, cell-cell communication can be measured by injecting a cell with one of the microelectrodes and the adjacent cells (with presumed shared gap junction channels) with the other microelectrode. This method is readily applicable for tissue type with widespread cellular communication through gap junctions \cite{penn1966ionic}. It has been used to unveil differences between cancer and normal cells within liver tissues \cite{loewenstein1966intercellular, loewenstein1967intercellular}. However the disadvantage of measuring electrical signals between cells is that they can be highly tissue specific (i.e. liver, heart and lens tissues) \cite{gros1983comparative}. Therefore measuring cell surface electrical signal has become less popular in recent years. 

In fact, cell-cell interaction within tissue context is a multi-dimensional problem. An intuitive way to capture cell-cell signalling is through an imaging approach. The development of higher resolution imaging technologies allows the studies of the anatomical organism and molecular interactions within experimental model organisms \cite{osswald2013insights}. Either confocal fluorescence or two-photon fluorescence microscopy can capture high detail images of intracellular structures in a cell. While a specific review of these two types of microscopy is beyond the scope of this report, it is worth noting that they both use laser light to excite fluorophores, with the resulting fluorescence captured by detectors. A great feature of fluorescence imaging is the ability to monitor a number of specific probes simultaneously, thus allowing co-localization or multiparameter imaging of structures within cells or tissues \cite{periasamy2013methods}. There exist a number of fluorescently tagged protein biosensors, which can be used to track the movement of some ligand molecules in the studies of cellular and molecular interaction in healthy and diseased tissues \cite{gerdes2013cell}. Although a number of studies have been done to model and quantify the activity of ligands and/or receptors \cite{awaji1998real, go1997quantitative, maamra1999studies, sneddon2003activation, bohme2009illuminating} using imaging data, they have limited scalability. The number of target ligands and/or receptors that can be measured per tissue is limited by the antibodies or RNA probes that be used concurrently which will also be discussed in section \ref{section:Imaging_sequecing_review}. 

% Alternatively, there is an approach to unveil cell communication through RNA sequencing data. As mentioned in section \ref{subsec:CCC_overview}, there are many software packages that allow to infer potential ligand-receptor interactions from scRNA-seq data through cross-referencing with database. These packages take pre-processed scRNA-seq data as input with cells grouped into clusters by genes expression. To correlate ligands and receptors from scRNA-seq to intercellular signalling, these packages use information from public resources (e.g. KEGG Pathway database \cite{kanehisa2007kegg} or The Database of Interacting Proteins \cite{salwinski2004database}). Although the computational approaches of each method differs with different scoring systems to rank significance of predicted ligand-receptor pairs, they all return a reasonable number of ligand-receptor pairs per analysis. This allows researchers to infer intercellular communications using single cell RNA-seq data. However, the shortcoming of these tools is the lack of spatial context. As such, although scRNA-seq has been a useful approach to reveal diversity of cellular population in cancer tissue for many years now, the loss of spatial information limit the use of these software packages in fully exploring cancer heterogeneity \cite{de2020unraveling}.

While in some cases a single technique might be sufficient, more often the combination of single-cell multi-omics and imaging technologies are required to solve the complexity of cell communication in cancer (Fig.\ref{fig:multimodal_approach_cci}). In addition, we need more advanced technologies which allow the acquisition of molecular profiles from single cells without the need of dissociating them from their tissue context \cite{de2020unraveling}. In fact, there are a few technologies which have been developed to enable single cell omics profiling without dissociation, many of them will be discussed in the following sections \ref{subsec:RNA_ISH}, \ref{subsec:ST_seq} and \ref{subsec:Spatial_proteomics}.

\subsubsection{Imaging approaches with immunofluorescence and immunohistochemistry}
As discussed above, an important feature of the cellular phenotype is its morphology. While scRNA-seq data can provide information about gene expression in cells, cells\' morphology is about cell physical feature such as locations, area, perimeter, solidity, eccentricity and circularity. Both gene expression and cell morphology are important features, especially in a disease context. The emergence of multiplexed spatial analysis enables dissecting the tumour microenvironment at multiple levels and higher granularity. In this section, I will review the most popular imaging and spatial sequencing technologies currently used to study tissue microenvironments and cell communication. Many of these technologies are able to assess transcriptomes and proteomes at the near-single cell resolution and are used in translational research with promising clinical applications. 

IHC is an umbrella term for the techniques that use antibodies or enzymes to identify specific proteins (antigens markers) in cells within tissue sections. The antibodies or enzymes are highly specific and only bind to the targeted proteins. There are many different ways to perform visualization of targets in tissues with IHC or IHC-based methods. However, based on the experimental settings, IHC can be categorized into two groups: colorimetric and fluorescent. In colorimetric IHC, the antibodies or enzymes bind to proteins and trigger colored precipitation which becomes visible to standard light microscopy \cite{BOURGEOIS2014132}. The enzymatic reaction or colored precipitation keeps producing products as long as there is no more regent or physical space to deposit more reactant \cite{corthell2014basic}. Consequently, the colorimetric staining IHC permanent in the tissue section and can be used for qualitative analyses where quantification is not considered important \cite{seidal2001interpretation}. 

Visualization of antigens by fluoresphore-conjugated antibodies is also often referred to as IF \cite{joshi2017immunofluorescence}. Unlike colorimetric IHC, fluorescence IF is not stable over long times and can only be observed after excitation with specific wavelengths \cite{corthell2014basic}. The antibody, when excited emits light of a different wave-length. There are two different fluorescence assays available for detecting antigens which are categorized by whether single antibody or two antibodies are used, namely direct and indirect IF \cite{JOSHI2017135}. The former protocol is simpler and used for labelling abundant target protein with a primary protein carrying a fluorophore that binds to a specific antigen. Meanwhile, the indirect IF protocol requires multiple stages of incubation; a primary antibody which binds only to the target and a secondary antibody with fluorophores to attach to primary antibody. The primary antibody in indirect IF can allow multiple secondary antibodies to bind to it, creating signal amplification effects. Therefore, indirect IF is better choice for detecting low-abundance targets. In comparison to colorimetric IHC, the benefit of using fluorophores is that they label multiple targeted proteins in the same tissue sample. However, unlike colorimetric staining, fluorescent labeling is not permanent. Since the fluorescent signal generally reflects the concentration of bound antibody \cite{dabbs2017diagnostic}, IF is considered a better choice for a quantitative experiment.

Due to relatively low costs, IHC and IF have been playing a central role in the visualization and identification of tissue antigens as well as in clinical diagnosis and prognosis for years \cite{ducheyne2015comprehensive, rupprecht2015current}. Depending on the purpose of the experiments and tissue preservation technique, IHC or IF be used interchangeably. While IHC studies are routinely used for pathological clinical diagnosis, the IF technique is the method of choice when an experiment requires investigation of colocalization of multiple proteins \cite{joshi2017immunofluorescence}. IHC and IF can be done on fresh frozen tissue yet most IHC is performed with Formalin-Fixed Paraffin-Embedded (FFPE) tissue. An example of the application of immunostaining in a clinical context is the used of IHC to evaluate the presence of HER2 in breast cancer sections. By using IHC and antibodies to detect HER2 receptor, clinicians can diagnose the current status of the patient to better decide on treatment options. Clinically, immunostaining is used for histopathology for the diagnosis of specific types of cancers based on known molecular markers.  

The results of IHC and IF heavily rely on the antibodies of choice and the capabilities of the technician performing them. Many studies highlighted that the choice of the antibody panel and the interpretation of the reaction patterns are the most important factors for  clinical outcome \cite{de2010immunohistochemistry, jensen1997immunohistochemistry}. In laboratory science, immunostaining by IHC and IF are limited by the number of protein markers it can feature at one time. Thus, IHC and IF are still being optimised for better performance in a multiplexed manner \cite{joshi2017immunofluorescence}. Meanwhile, IF also inspired the use of fluoresphore-conjugated probes to directly visualize specific DNA and RNA sequences in the tissue which is often referred to as fluorescent in situ hybridization (FISH). In addition to more recent advances in IHC and IF, there are now several new protein imaging technologies based on mass spectrometry that allow the multiplexing of numerous proteins (up to 100 protein markers) with huge potential applications in research especially for cell colocalization and interaction studies within tissues. The availability of these imaging and staining technologies might gradually replace the position of IHC and IF in laboratory research. However, until the new technologies are mature enough, clinicians still relying heavily on the use of IHC and IF. 

\subsection{Spatial transcriptomics}
% overview of spatial omic data, 
\subsubsection{Spatial transcriptomic with ex-situ sequencing}
\label{subsec:ST_seq}
The widespread use of single-cell data over the past few years in form of single cell RNA sequencing (scRNA seq) has unveiled rare cellular functions and biologically meaningful cell-to-cell variability. However, in order to perform single-cell analyses for solid tissues, cell dissociation is required. The dissociation procedure causes loss of cell positional information, making it challenging to link the transcriptomes back to their original spatial location. The lack of spatial context can be challenging when studying cell-cell interaction especially in cancer. For that reason, a new field called \textit{spatially resolved} transcriptomics is emerging and aiming to capture gene expression profile while retaining information of the tissue context \cite{burgess2019spatial}. 

While RNA ISH can be considered as subclass of spatial transcriptomic, it has a very limited number of transcripts that can be captured simultaneously and can be only applied to known markers. A different approach is to directly sequence the transcripts while they are still in the tissue, which is called \textit{in situ} sequencing (ISS). Some ISS technologies can overcome the low plex level of RNA ISH like the Transcript Amplicon Readout Mapping (STARmap) technology can capture up to 1020 genes a the tissue \cite{wang2018three}. Yet, that approach still requires probe hybridization, and shares some limitations of RNA ISH \cite{ke2013situ,hernandez2019mapping,chen2018efficient} regarding scalability to include unknown genes. Moreover, the costs profile higher number of target genes become too high to be feasible outside of well financed the science laboratories.

Another concept that seems currently to be the most promising technology for spatial transcriptomic is to spatially profile the complete transcriptome by capturing and barcoding the transcripts \textit{in situ} followed by \textit{ex situ} sequencing \cite{asp2020spatially}. There are many experimental methods that have implemented spatially resolved transcriptomics through this concept. Since the first one called Spatial Transcriptomic (ST) was established in 2016, there have been a series of new technologies introduced including Slide-Seq, high-definition spatial transcriptomics (HDST), NanoString GeoMx in 2019. The RNA capture efficiency can greatly effect the quality of these methods, especially at higher resolution. Since most of the current existing technologies for \textit{ex situ} spatial transcriptomics sequencing (ST—seq) are new, they require optimization. However, while the resolution currently only ranges from 2$\mu m$ — 100 $\mu m$ which means the capture rate is close to fewer than 40 cells, the experimental costs are considerably lower than for ISS. In addition, as ST-seq sequences full transcriptome, it likely will be the first technology that can be applied in clinical practice or medical translational studies.                
Among the technologies to perform ST-seq. ST was acquired by the company 10X Genomic in 2018 which is now providing commercial kits. The first version of ST publish in 2016 used a surface glass slides which was printed with barcoded probes grouped into circle spots. Each spot has specific barcode id that is unique to the x and y location on the glass slide. The tissue is fixed, imaged, and permeabilized on top of the spots. During the permeabilization process, the mRNAs defuse out of the tissue and are captured by the probes. The final step is to reverse transcribe the mRNA and extract the barcoded cDNA-mRNA to sequence \cite{staahl2016visualization, berglund2018spatial}. With latest release of the Visium Spatial Transcriptomic protocol (10X Genomics), the diameter of barcoded spots is reduced from  100$\mu$m to 55 $\mu$m with smaller distance between the spots. While both  Visium  and ST are currently limited to fresh frozen tissues, they have already been used for the analysis of the tissue heterogeneity and gene expression in tumour tissues \cite{berglund2018spatial, thrane2018spatially, moncada2019integrating,ji2020multimodal, yoosuf2020identification} and inflammatory tissues \cite{carlberg2019exploring}. Results from the studies applied to tumour tissues once again confirmed the role of the microenvironment in promoting tumour progression \cite{thrane2018spatially, moncada2019integrating}. Additionally, corresponding histopathology images of tissues stained with haematoxylin and eosin (H\&E) coupled with ST allow the application of deep learning approaches for integrative analysis of histology images with the ST-seq profile \cite{he2020integrating, tan2019spacell}. Currently, protocols to enable recovery of mRNA from FFPE tissue sections for ST or 10X Visium are being extensively developed. As FFPE is the preferred technology to preserve clinical biospecimens, I believe that there will be more extensive analysis of tumour tissues
once the protocols have become more matured.           

Given the complexity and heterogeneity of tumours, the untargeted methods like Visium 10X  ST, NanoString GeoMX can extend our understanding of immune cell infiltration both in localized and metastatic disease. In comparison to RNA ISH or ISS, ST-seq provides relatively lower degree of spatial resolution. However, with the advantage of the whole transcriptome profile, ST is undoubtedly a powerful and promising complementary technology to study cell communication within tissue context.  

\subsubsection{RNA in situ hybridization (ISH)}
An interesting analogy for detecting a specific DNA sequence on the chromosomes or in a cell is like looking for a needle in a haystack. The needle is being the DNA sequence of interest and, the haystack representing all chromosomes. This search is made much easier if the investigator has a powerful "magnet"—in this case, a fluorescent copy of the DNA sequence of interest \cite{Connor2008natureEdu}. While IHC identifies proteins in tissue, RNA in situ hybridization is the technique to identify where RNAs are present in the tissue. The first attempt to localize mRNA was conducted in 1969 and used radioactive labels and hybridization probes in the nuclei of frog eggs. To this day, radioactive probes are still available and seem to be the most sensitive choice. However, the high cost of the experiment and potential hazard make radioactive approaches less favourable. Soon after the first experiment with radioisotopes, fluorescent labels make great strides in replacing radioactive probes because of safety profile, stability, and ease of detection \cite{rudkin1977high, Connor2008natureEdu}. Since then, numerous RNA ISH methods based on the FISH approach have been developed, including single molecule FISH (smFISH), RNAscope, sequential FISH (seqFISH), multiplexed error-robust FISH (MERFISH), cyclic-ouroboros smFISH (osmFISH), and most recently DNA microscopy. 

Conceptually most of the current developed technologies for RNA ISH are FISH-based and use fluorescently labeled probes and that are hybridized to predefined RNA targets in order to visualize the presence of transcripts. Probes are either directly or indirectly hybridized to the molecules and subsequently visualized by microscopy. A benefit of RNA ISH is the ability to characterise the presence of RNA sequences within the tissue without cells dissociation. Although the basic principle of FISH remains unchanged, the sensitivity and multiplexing capabilities have been advanced considerably. The first attempt to increase the sensitivity of FISH protocol was in 2012 with the release of smFISH (single molecule FISH). By using multiple short oligonucleotide probes (20-50 pb) to multiple regions of a transcript, smFISH produces higher and more robust signal. Nowadays, there are improved versions of smFISH, including seqFISH and MERFISH, which allow researchers to target more transcripts at a time through serial rounds of hybridization, imaging and probe stripping \cite{asp2020spatially}. While these methods have the advantage of multiplexed detection (ranging from 8 to 20 targets), they usually improve the sensitivity of ISH by adopting signal amplification of either target nucleic sequences prior to ISH (e.g in situ PCR) or signal detection after the hybridization process \cite{qian2003recent}. The signal amplification creates quantification problem for RNA ISH, and limits the application of RNA ISH in clinical analysis \cite{levsky2003fluorescence,wang2012rnascope}. There are other RNA ISH technologies that can perform independent signal amplification such as RNAscope and DNA microscopy. While DNA microscopy is an optic-free mapping of nucleotides and does not rely on the physical location of cells, it is still a very new technology and has only been applied to a small subset of transcripts \cite{asp2020spatially,weinstein2019dna}. Consequently, RNAscope technology might be the better option for FISH which allows signal amplification and background suppression by itself.

RNAscope assay is designed with ``Z-like" probes where the lower region of the Z is (18-25pb) complementary to the target RNA and the upper region (14 bp) forms the binding site for the fluorescently label probe. Two assay probes with ZZ pairs are designed to bind in a region spanning 1000 bases of target RNA, this allows the specific detection and signal amplification from each target RNA molecule \cite{solanki2020visualization}. As a result, the RNAscope assay is best used for RNA sequences with more than 300 nucleotides. For short RNA sequences, a variant of RNAscope called BaseScope is a more suitable option. By using probe specific amplifiers and sequence labelling, this assay can visualize multiple target RNAs simultaneously. Originally, RNAscope could only capture 4 different types of transcripts per experiment due to a limited number of spectrally discernible fluorescent dyes \cite{wang2012rnascope}. However, the latest version released 2019, the Hiplex RNAscope assay has increased the number of RNA targets to twelve. In comparison with the antibody-based IHC assay, RNAscope assays have better multiplexing capability than an antibody-based approach for protein. As RNAscope technology can produce more sensitive results and achieves higher levels of signal amplification, the technology is potentially compatible with clinical routine.

The advancement of RNA ISH methods has not only allowed us to integrate the cells functions with spatial information but also has expanded our understanding of the structural organization of normal and pathological tissues. While RNA in situ hybridization is becoming more popular in basic research, the use of that technologies in clinical routine is quite limited to highly expressed genes (e.g EBER1/2 in EBV-related diseases) \cite{gulley2001molecular}. The reason for its limited application in clinical diagnostics stems from the high technical complexity and the insufficient sensitivity of many the RNA ISH technologies. Like IHC and IF (which can be considered as protein in situ) RNA ISH can also be potentially integrated into the clinical practice when the protocols are better optimized and the technology is able to detect more different gene transcripts.

\subsubsection{Spatial proteomics}
Similar to spatially resolved transcriptomics, spatial proteomics refers to the set of methods that use imaging technology for the visualization of proteins in their native cellular environment without the need for physical separation of cells or organelles before proteomic analysis. In addition to understanding the morphological context of RNAs, the spatial distribution of proteins is also equally important in studying cancer. The protein localization directly connects to protein function in health and disease. Being able to capture the localization of proteins and their dynamics at the cellular level is essential for a complete understanding of cell biology \cite{lundberg2019spatial}. 

Recent developments in flow cytometry, high-throughput microscopy and mass spectrometer have enabled to apply conventional protein study strategies in a spatially resolved context. The first wave of spatial proteomics methods used mass cytometric (MC) analysis. As an off-shoot from IHC/IF technology, MC-based spatial proteomic methods use antibodies to visualize the protein. An advantage of this type of analysis approach is that it can easily be used to study proteins in many different cell types. Two notable technologies have been developed that are based on the MC approach including Hyperion Imaging Mass Cytometry (IMC) as the pioneer and Multiplex Ion Beam Imaging (MIBI) as the direct competitor platform. Both of the methods use antibodies conjugated to stable metal isotopes. In IMC, after the tissue section is immobilized on slides. it is stained with panels of antibodies then it enters a laser ablation chamber where it is rasterized until it plumes. The aerosolized/ionized plumes of tissue are fed into time-of-flight mass spectrometer for analysis of isotope abundance.  There was a study that combined two technology of RNAscope and IMC workflow to interrogate the interplay between transcription and protein in the tissues \cite{schulz2018simultaneous}. Instead of using fluorophores, the RNAscope probes is modified to tag with metal. Thus mRNA and protein can be simultaneously measured in single cells. By using pure and rare-earth element isotopes, IMC can overcome the limitations of spectral overlap observed in IF and allow to quantify up to 38 tagged channels simultaneously. However, the use of lasers damages the tissues that are being scanned. MIBI can be used as an alternative approach to imaging histological sections labeled with isotope-tagged antibodies. Most of the steps for measuring protein levels using MIBI are quite similar to IMC. However, in MIBI secondary ion mass spectrometry is needed and  an (oxygen) ion beam is used instead of laser ablation to raster over the tissue. Subsequently, ions are analysed using mass spectrometer. While the MIBI only ablates a thin layer of the tissue (20-50nm) and cause fewer damages, the technology creates matrix effects and makes quantification more challenging \cite{bodenmiller2016multiplexed}. [...]

There exists another branch of technology called Mass Spectrometry Imaging (MSI) that characterises protein expression using mass spectrometer. Matrix‐Assisted Laser Desorption/Ionization (MALDI) tissue imaging is a technology using this approach and has been proven to be more versatile than MC. In MALDI, a laser and mass spectrometer are used to ablate the tissue and ionize the molecules. Eventually, the ionized molecules are fed to the spectrometer to determine the molecular weights \cite{caprioli1997molecular}.  This is performed in a label-free manner to measure the proteins present in the tissue. In this way MALDI is more suitable for the non-targeted experiments. However, there are also a number of limitations for MALDI or MSI in general such as lower spatial resolution, reduced sensitivity for larger protein, and low multiplexing capability. [...]

It is worth noting that protein visualization using fluorescence-based staining remains as another option. The advantages of using fluorescently conjugated antibodies to capture the presence of proteins are inherited from IHC and IF. These approaches involve iterative rounds of staining the protein with fluorescent probes which subsequently are detected by specialized instruments. There are multiple technologies for multiplexing fluorescence-based protein staining including NanoString GeoMx and co-detection by indexing (CODEX). In Nanostring DSP, a multiplexed cocktail of primary antibodies with  unique ultraviolet-photocleavable DNA oligos is used to target the proteins. After exposure to ultraviolet light from the GeoMX instrument, pools of released indexing oligos were hybridized to the optical barcodes and then read by another instrument (e.g NanoString nCounter or Illumina MiSeq) \cite{de2020unraveling, helmink2020b}. This enables to simultaneously detect up to 96 proteins and over 1000 RNA targets. CODEX have been developed by the inventor of MIBI and also uses a specialized instrument to convert images from fluorescence microscopes into a highly-multiplexed imaging systems. CODEX currently allows the detection of over 60 protein markers within the tissue by iteratively staining and scanning the tissue sections\cite{goltsev2018deep}. A disadvantage of CODEX, but also many other FISH-based technologies and IMC, is lack of signal amplification which results in under-detection of lowly abundant protein/gene markers. A comparative overview for three different approaches to perform spatial proteomic is available in Table \ref{table:SpatialProteomicComparison}. [...]

\section{Research Objectives}
text
\section{Thesis Overview}
text

\bibliography{Chapter1}