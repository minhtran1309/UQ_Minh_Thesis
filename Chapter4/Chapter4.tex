\chapter[The association between tumour spatial structure and clinical diagnostics]{The association between tumour spatial structures and clinical diagnostics or subtype classification}
\label{Chap:4}	%CREATE YOUR OWN LABEL.
\pagestyle{headings}
\label{Sec:4.1_intro}	%CREATE YOUR OWN LABEL.
The tumour microenvironment is a dynamic ecosystem with diverse cells self-organise and complex networks of cell-cell neighborhood. Using communities as the representation of different regions within the tissue, we aim to develop quantitative computation approaches to infer differences between difference cancer subtypes and clinical outcome.     


\section{MOSAP - MultiOmics Spatial Analysis Platform}

\subsection{The need for multiomics spatial analysis}


\subsection{MOSAP methods and software}


\section{The variation of tumor microenvironment structures in skin cancer}
The spatial distribution of cancer cells in skin cancer have distinct enrichment patterns across different subtype including BCC, SCC, melanoma. 

\subsection{The variation of tumour heterogeneity across skin cancer subtypes}
Here we measure the variation of population of different cell types and cell type variation from multiple FOVs across the patient tissues to find (1) spatial patterns enrich for each skin cancer subtypes, (2) the gene markers that enrich for specific communities. 

Potential approach to cancer marker screening 
\subsection{Conserved tumor microenvironment communities across skin patient samples}
Result with the comparison of cell type across multiple Polaris, CosMx images 

\section{The tumor microenvironment analysis across patients in colorectal cancer samples}
\subsection{TME variation as a potential predictive model for clinical outcome}

% % \label{Sec:4.2_Cell_communities}	%CREATE YOUR OWN LABEL.
% \subsection{The correlation of genomic features and cell types neighborhood}
% \subsection{The spatial features of cell-cell interaction as predictive model of survival rates}
% ********* Enter your text below this line: ********

% ***************************************************


% ***************************************************
% \section{}
% \label{Sec:4.3_quantitative_validation}	%CREATE YOUR OWN LABEL.
% \subsection{}
% ********* Enter your text below this line: ********

% ***************************************************
% \bibliographystyle{elsarticle-num}

% \bibliography{./References/Bibliography}