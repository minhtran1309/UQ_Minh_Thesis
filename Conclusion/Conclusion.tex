% ***************************************************
% Conclusion
% ***************************************************
\chapter[Conclusion and future perspective]{Conclusion and future perspective}
\label{Chap:Conclusion}
\pagestyle{headings}
\section{Conclusion}
% ********* Enter your text below this line: ********
In this thesis, I have developed several analysis pipelines for detecting and validating cell colocalisation and neighbourhoods from spatial transcriptomic and proteomic data. In Chapter 2, an analysis pipeline called Spatial TRanscriptomic \textit{In Situ} Hybridization (STRISH) was established to study cell colocalisation through ligand-receptor (L-R) pairs in cancer across whole tissue sections. The STRISH pipeline utilised two complementary sequencing technologies to discover the L-R pairs that cells could use for interaction, including single-cell RNA and spatial transcriptomic. The L-R pair were validated using the targeted RNAscope and/or multiplexed immunofluorescence Opal Polaris. We applied the STRISH pipeline to study in-depth two of the most common skin cancer types (Basal Cell Carcinoma and Squamous Cell Carcinoma - BCC, SCC). The analysis pipeline demonstrated the feasibility of discovering the L-R pairs through genome-wide approaches, followed by targeted validation when applied to the study of cell colocalisation in our skin cancer dataset. 

In Chapter 3, we shifted the focus to spatial proteomics, which adds another layer of information to provide a holistic view of cells in their spatial context. Two representatives of spatial proteomic technologies were applied to capture protein expression from specimens of human skin cancer and colorectal cancer, including multiplexed Opal Polaris and Imaging Mass Cytometry, respectively. We developed comprehensive pipelines for spatial analysis of the two datasets. Using spatial analysis, we were able to study the tumour microenvironment through cell community detection and cell type co-occurrence from spatial proteomic data. Additionally, we demonstrated the versatility of the analysis methods by extending the applications to study the cellular interaction in SARS-CoV-2 infected samples. Overall, the analysis methods developed in Chapter 3 have shown robustness and produce interpretable results when applied to the study of cellular colocalisation from individual samples of spatial proteomic data.

One of the challenges in using spatial omics methods to characterize cell types is the limited number of features measured at each location using either spatial transcriptomic or proteomic. A strategy to overcome this limitation is to combine multiple spatial-omic technologies and carry out the experiment through multiple samples. In Chapter 4, we developed the Multi-Omics Spatial Analysis Platform (MOSAP) to effectively address the challenges associated with multi-sample and multimodal analysis. The key spatial-omic data integration is a well-chosen set of overlapping biomarkers to join the samples from different spatial technologies. Given the throughput of the spatial technologies may be different to each other, the list of biomarkers that we can select for the experiment would hinder the data integration. Therefore, we introduced the $Mosadata$ object, a data object within the MOSAP that was specifically designed to handle the dissimilarity of biomarkers multi-sample and/or multimodal data processing. Multiple samples can be stored in a single $Mosadata$ object and interact with each other through the MOSAP analysis features. The robustness of the MOSAP platform was demonstrated through the quantitative analysis of two distinct types of spatial-omic data, namely CosMx for spatial transcriptomics in the skin cancer dataset and Imaging Mass Cytometry (IMC) for spatial proteomics in the colorectal cancer (CRC) dataset. The platform enabled valuable insights into the complex cellular composition and heterogeneity of the two datasets and facilitated the comparison of intra-tumour heterogeneity and inter-tumour comparison.  

Overall, the analysis presented in this thesis leveraged both imaging and sequencing components of spatial-omics technologies to provide a comprehensive view of cellular communication, including paracrine and juxtacrine signalling. The STRISH pipeline was established to study cancer and immune cell interactions, employing genome-wide approaches for initial discovery and targeted validation for confirmation. Chapter 3 expanded on spatial analysis to shed light on how different cell types in the cancer tissue colocalised and formed the community of cells. This chapter also demonstrated the adaptability of the spatial proteomic data analysis pipeline when applied to study the spatial organization of immune cells in Covid-infected tissues. Finally, in Chapter 4, the MOSAP was developed, incorporating multiple spatial neighbourhood networks and spatial heterogeneity scoring functions, specifically designed for multi-sample analysis and data integration across multimodal experiments. The image registration of the MOSAP enabled the graphical user interaction component within napari, which hopefully can attach users without programming experience. These achievements collectively contribute to a deeper understanding of cell-cell interactions at cellular resolution and inter-tumour microenvironment heterogeneity of various cancer types. 
\section{Future perspective}
The continued expansion of spatial-omics experiments is set to, either significantly enhance the number of biomarkers throughput that it can capture or increase the sensitivity of the imaging process. Furthermore, given the increasing number of research studies adopting spatial-omics data, the improved computational methods will also continue to push the adoption of spatial profiling of different -omics even further. Two key components are anticipated to drive the future development of the spatial-omics area: computational analysis tools and experimental methodologies. 

Regarding the experimental development, 

Application of spatial-omics in clinical research is still hindered by the cost and profession required by the technologies. However, the last few years witnessed an unprecedented number of publications that adopted spatial-omics technologies to study cancer (\ie, prostate, melanoma, breast and colorectal cancer \cite{berglund2018spatial, ji2020multimodal, thrane2018spatially, schurch2020coordinated, ji2012single}).  


 \cite{bankhead2017qupath,schneider2012nih}.  



% ***************************************************
% \bibliographystyle{elsarticle-num}

% \bibliography{./References/Bibliography}