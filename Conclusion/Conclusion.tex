% ***************************************************
% Conclusion
% ***************************************************
\chapter[Conclusion and future perspective]{Conclusion and future perspective}
\label{Chap:Conclusion}
\pagestyle{headings}
% ********* Enter your text below this line: ********
In this thesis, I have developed a pipeline for detecting and validating cell colocalisation through ligand-receptor pair using spatial transcriptomics technologies (Chapter \ref{Chap:2}). Next, I demonstrated the analyses of spatial proteomics in Chapter 3. The analyses utilised the spatial coordinate of multiplex Imaging Mass Cytometry (IMC) and multiplexed Vectra Polaris  (Chapter 4).

Multiple disorders are associated with cancer, including genetic instability and environmental factors. However, the central issue that hinders successful anti-cancer treatment is the variability of cell organisations across TMEs and patients. Studies have shown that TMEs are shaped and coordinated by the crosstalk between cancer cells to facilitate the development of cancer \cite{jin2020updated, quail2013microenvironmental, schurch2020coordinated}. Spatially, TMEs are often diversely organised and constantly change in response to external stimulations (i.e. chemotherapy, radiotherapy), which ultimately assist these cells’ survival and metastasis to neighbour organisms \cite{wu2022spatial}. Experimental and computational approaches are highly desirable to properly understand the heterogeneity and mutability of the TMEs across multiple samples and cancer subtypes. Using spatial multi-omics profiles could enable us to shed light on fundamental processes of tumourigenesis and develop suitable treatments.  
% check https://www.biorxiv.org/content/10.1101/2020.05.28.122614v1.full.pdf

[Microenvironment regulation of tumour progression and metastasis ] 
 

In comparison, it is generally agreed that chronic inflammation caused by immune response is a pro-tumorigenesis incident. It is also critical to note that impaired immune responses can correlate with high cancer incidence. In a statistical analysis of up to 26,000 female patients receiving the immunosuppressant post-organ transplant, the tumour incidence was higher than predicted for multiple cancers, including lung, gastrointestinal, reproductive and skin cancers \cite{stewart1995incidence}. In contrast, breast cancer incidence decreased in this cohort, illustrating the paradoxical nature of immune responses.   Similar retrospective analyses in immunodeficiency syndrome patients have also indicated that adequate immune function may be protective against certain cancers \cite{quail2013microenvironmental}.  


In recent years, a number of graphical user interface (GUI) open-source packages specialised for multiplexed imaging have been developed. These open-source packages allow users to contribute to the analysis power through plugins, scripts, or pipelines \cite{bankhead2017qupath,schneider2012nih}.  
 
% 


Conclude your thesis.
\subsection{Discussion}
\subsection{Future direction}
% ***************************************************
% \bibliographystyle{elsarticle-num}

% \bibliography{./References/Bibliography}